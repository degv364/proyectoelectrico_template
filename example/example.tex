% Created 2018-05-07 Mon 17:15
% Intended LaTeX compiler: pdflatex
\documentclass{proyectoelectrico}
\usepackage[utf8]{inputenc}
\usepackage[T1]{fontenc}
\usepackage{graphicx}
\usepackage{grffile}
\usepackage{longtable}
\usepackage{wrapfig}
\usepackage{rotating}
\usepackage[normalem]{ulem}
\usepackage{amsmath}
\usepackage{textcomp}
\usepackage{amssymb}
\usepackage{capt-of}
\usepackage{hyperref}
% -----------------------------------------
% OTROS PAQUETES E INSTRUCCIONES ESPECIALES
% -----------------------------------------

\usepackage[utf8]{inputenc}
\usepackage[T1]{fontenc}
\usepackage{svg}
\usepackage{graphicx}
\usepackage{grffile}
\usepackage{longtable}
\usepackage{wrapfig}
\usepackage{rotating}
\usepackage[normalem]{ulem}
\usepackage{amsmath}
\usepackage{textcomp}
\usepackage{amssymb}
\usepackage{capt-of}
\usepackage{hyperref}
\usepackage{systeme}
\usepackage{url}



% Para insertar código fuente estilizado
\usepackage{listings}
	\lstset{basicstyle=\ttfamily,
    		breaklines=true,
            numbers=left, 
    		numberstyle=\tiny, 
    		stepnumber=1, 
    		numbersep=6pt}

% Para insertar símbolos extraños
\usepackage{marvosym}

% Para insertar texto fútil
\usepackage{lipsum}

% NUEVAS INSTRUCCIONES

\newcommand{\EIEx}{\textsc{Escuela \Lightning~ Ingeniería Eléctrica}}

% Definición de algunos símbolos matemáticos
\newcommand{\me}{\mathrm{e}}
\newcommand{\mi}{\mathrm{i}}
\newcommand{\mj}{\mathrm{j}}
\newcommand{\md}{\mathrm{d}}
\titulo{Trying not to use LaTeX}
\autor{Daniel García Vaglio}
\carne{B00000}
\guia{Dr. rer. nat Guia}
\lectorA{Lector, Dr Ing}
\lectorB{Lector, PhD}
\mes{7}
\ano{2018}
\date{\today}
\title{}
\hypersetup{
 pdfauthor={},
 pdftitle={},
 pdfkeywords={},
 pdfsubject={},
 pdfcreator={Emacs 25.3.1 (Org mode 9.1.3)}, 
 pdflang={English}}
\begin{document}

\begin{LaTeX}
\frontmatter
\portada
\aprobacion

% EL RESUMEN
% ----------

\begin{resumen}{palabras, claves, separadas, por, coma}

Este es el resumen del trabajo. Tiene un máximo de 300 palabras y debe ajustarse a una sola página en este formato.

\lipsum[1-2]

\end{resumen}
% EL RESUMEN EN INGLÉS
% --------------------

\begin{theabstract}{Title of the Project in English}{keywords, separated, by, a, comma}

This is a test of the abstract of the project. It should not have more than 300 words or exceed one page in this template (whichever happens first).

\lipsum[1-2]

\end{theabstract}

% El entorno 'theabstract' tiene el formato \begin{theabstract}{A} ...B... \end{theabstract} donde A es el título del proyecto traducido de inglés a español y B es el contenido, en inglés, del resumen. Se recomienda buscar ayuda calificada para la elaboración y/o revisión de este resumen.

% LA DEDICATORIA
% --------------

\begin{dedicatoria}
Dedicado a
\end{dedicatoria}
% LOS AGRADECIMIENTOS
% -------------------

\begin{agradecimientos}

\lipsum[5]

\end{agradecimientos}

\tableofcontents
\listoffigures
\listoftables

\begin{itemize}
\item x es un vector canónico
\end{itemize}

\mainmatter
\end{LaTeX}

\chapter{Introducción}
\label{sec:org771cc48}

Esto es un texto introductorio. Se debe escribir bien porque la gente decide si le interesa o no el
proyecto con esta parte.


\section{Objetivos}
\label{sec:org8731f97}
\subsection{Objetivo General}
\label{sec:orgf205699}

Quiero dejar de utilizar \LaTeX{} para hacer mi proyecto eléctrico y usar algo más cómodo y moderno

\subsection{Objetivos específicos}
\label{sec:org2e712d8}
\begin{itemize}
\item Org es mejor que latex
\item Latex no es mejor que org
\end{itemize}

\section{Metodología}
\label{sec:org1552d28}
Buscar información sobre org y latex para poder resolver este asunto.

\chapter{Marco Teórico}
\label{sec:orgd4060a0}
Esto no tiene mucha teoría que digamos, solo voy a dejar unas citas de lo que he usado. La plantila
está disponible en la página del curso, si no tiene acceso no se preocupe, aquí está todo lo
necesario. Esto sirve en Emacs, esta es la wiki \cite{emacs_wiki}. La documentación de ORGmode está
aquí \cite{org_wiki}. Para las citas uso ORG-ref, muy útil, pueden encontrarlo aquí \cite{org_ref}

\chapter{Desarrollo}
\label{sec:org8bc4846}
\chapter{Resultados}
\label{sec:orga2c7b2a}
\section{Trabajo futuro}
\label{sec:org9cad9ef}

\begin{LaTeX}
\appendix
\end{LaTeX}
\chapter{Apéndice}
\label{sec:org7e2aa08}
Esto es una apéndice duh



\bibliographystyle{plain} 
\bibliography{biblografia}
\end{document}